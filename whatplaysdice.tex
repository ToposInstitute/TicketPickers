\documentclass[11pt]{article}

\usepackage{amsthm}
\usepackage{mathtools}

\usepackage[inline]{enumitem}
\usepackage{ifthen}
\usepackage[utf8]{inputenc} %allows non-ascii in bib file
\usepackage{xcolor}

\usepackage[backend=biber, backref=true, maxbibnames = 10, style = alphabetic]{biblatex}
\usepackage[bookmarks=true, colorlinks=true, linkcolor=blue!50!black,
citecolor=orange!50!black, urlcolor=orange!50!black, pdfencoding=unicode]{hyperref}
\usepackage[capitalize]{cleveref}

\usepackage{tikz}

\usepackage{amssymb}
\usepackage{mathrsfs}
\usepackage{dutchcal}
\usepackage{fontawesome}
\usepackage{stmaryrd}

% cleveref %
  \newcommand{\creflastconjunction}{, and\nobreakspace} % serial comma
  \crefformat{enumi}{\card#2#1#3}
  \crefalias{chapter}{section}


% hyperref %
  \hypersetup{final}

% enumitem %
  \setlist{nosep}
  \setlistdepth{6}



% tikz %



  \usetikzlibrary{ 
  	cd,
  	math,
  	decorations.markings,
		decorations.pathreplacing,
  	positioning,
  	arrows.meta,
  	shapes,
		shadows,
		shadings,
  	calc,
  	fit,
  	quotes,
  	intersections,
    circuits,
    circuits.ee.IEC
  }
  



%-------- Renewed commands --------%

\renewcommand{\ss}{\subseteq}

%-------- Other Macros --------%



\DeclareMathOperator{\Hom}{Hom}
\DeclareMathOperator{\Mor}{Mor}
\DeclareMathOperator{\dom}{dom}
\DeclareMathOperator{\cod}{cod}
\DeclareMathOperator{\idy}{idy}
\DeclareMathOperator{\comp}{com}
\DeclareMathOperator*{\colim}{colim}
\DeclareMathOperator{\im}{im}
\DeclareMathOperator{\ob}{Ob}
\DeclareMathOperator{\Tr}{Tr}
\DeclareMathOperator{\el}{El}
\DeclareMathOperator{\Funn}{Fun}

\newcommand{\Set}[1]{\mathsf{#1}}%a named set
\newcommand{\ord}[1]{\mathsf{#1}}%an ordinal
\newcommand{\cat}[1]{\mathcal{#1}}%a generic category
\newcommand{\Cat}[1]{\mathbf{#1}}%a named category
\newcommand{\fun}[1]{\mathrm{#1}}%a function
\newcommand{\Fun}[1]{\mathit{#1}}%a named functor


\newcommand{\N}{\mathbb{N}}
\newcommand{\ox}{\otimes}
\newcommand{\Poly}{{\sf Poly}}

\newcommand{\id}{\mathrm{id}}
\newcommand{\then}{\mathbin{\fatsemi}}

\newcommand{\cocolon}{:\!}


\newcommand{\iso}{\cong}
\newcommand{\too}{\longrightarrow}
\newcommand{\tto}{\rightrightarrows}
\newcommand{\To}[2][]{\xrightarrow[#1]{#2}}
\renewcommand{\Mapsto}[1]{\xmapsto{#1}}
\newcommand{\Tto}[3][13pt]{\begin{tikzcd}[sep=#1, cramped, ampersand replacement=\&, text height=1ex, text depth=.3ex]\ar[r, shift left=2pt, "#2"]\ar[r, shift right=2pt, "#3"']\&{}\end{tikzcd}}
\newcommand{\Too}[1]{\xrightarrow{\;\;#1\;\;}}
\newcommand{\from}{\leftarrow}
\newcommand{\fromm}{\longleftarrow}
\newcommand{\ffrom}{\leftleftarrows}
\newcommand{\From}[1]{\xleftarrow{#1}}
\newcommand{\Fromm}[1]{\xleftarrow{\;\;#1\;\;}}
\newcommand{\surj}{\twoheadrightarrow}
\newcommand{\inj}{\hookrightarrow}
\newcommand{\wavyto}{\rightsquigarrow}
\newcommand{\lollipop}{\multimap}
\newcommand{\imp}{\Rightarrow}
\newcommand{\down}{\mathbin{\downarrow}}
\newcommand{\fromto}{\leftrightarrows}
\newcommand{\tickar}{\xtickar{}}
\newcommand{\slashar}{\xslashar{}}


\newcommand{\inv}{^{-1}}
\newcommand{\op}{^\tn{op}}

\newcommand{\tn}[1]{\textnormal{#1}}
\newcommand{\ol}[1]{\overline{#1}}
\newcommand{\ul}[1]{\underline{#1}}
\newcommand{\wt}[1]{\widetilde{#1}}
\newcommand{\wh}[1]{\widehat{#1}}
\newcommand{\wc}[1]{\widecheck{#1}}
\newcommand{\ubar}[1]{\underaccent{\bar}{#1}}



\newcommand{\bb}{\mathbb{B}}
\newcommand{\cc}{\mathbb{C}}
\newcommand{\nn}{\mathbb{N}}
\newcommand{\pp}{\mathbb{P}}
\newcommand{\qq}{\mathbb{Q}}
\newcommand{\zz}{\mathbb{Z}}
\newcommand{\rr}{\mathbb{R}}


\newcommand{\finset}{\Cat{Fin}}
\newcommand{\smset}{\Cat{Set}}
\newcommand{\lgset}{\Cat{SET}}
\newcommand{\smcat}{\Cat{Cat}}
\newcommand{\lgcat}{\Cat{CAT}}
\newcommand{\prof}{\mathbb{P}\Cat{rof}}

\newcommand{\set}{\tn{-}\Cat{Set}}
\newcommand{\coalg}{\tn{-}\Cat{Coalg}}
\newcommand{\const}[1][\blank]{#1 !}

\newcommand{\hh}[2][]{#1 \tn{\textit{#2}} #1}
\newcommand{\qqand}{\hh[\qquad]{and}}
\newcommand{\qand}{\hh[\quad]{and}}
\renewcommand{\iff}[1][\;\;]{#1\Leftrightarrow#1}
\newcommand{\ifff}[1][\;\;]{#1\xLeftrightarrow{\quad}#1}
\newcommand{\hi}[4][]{#1 #2 \tn{\textit{#4}} #3}
\newcommand{\where}[1][,]{\hi[#1]{\qquad}{\quad}{where}}
\newcommand{\qimplies}{\hh[\quad]{$\implies$}}

\newcommand{\lott}{\mathit{lott}}
\newcommand{\yon}{\mathcal{y}}


\newcommand{\dnote}[1]{{\quad \color{blue}$\lozenge$\;David says:}~#1\;{\color{blue}$\lozenge$}\quad}
\newcommand{\pnote}[1]{{\quad \color{red}$\lozenge$\;Priyaa says:}~#1\;{\color{red}$\lozenge$}\quad}


\title{The ``typical'' $[\lott,\yon]$-coalgebra}
\title{Does God play dice? Or...\\What's the best $[\lott,\yon]$-coalgebra?}
\author{
   David I.\ Spivak \quad \quad \quad Priyaa Varshinee Srinivasan}
\date{\vspace{-.1in}}

\begin{document}

\maketitle

\begin{abstract}

\end{abstract}

\section{Introduction}

Einstein famously said ``God does not play dice" in response to... quantum something. But also in computers we need to \emph{sample} from probability distributions. So what does play dice? In this post we'll explain how this question relates to another one: what is the best $[\lott,\yon]$-coalgebra? We'll explain how a $[\lott,\yon]$-coalgebra is a winning-ticket-picker for lotteries, and explain a few senses in which one may consider one the best!

\paragraph{Acknowledgements.} Thanks to Maxine Collard for useful conversations.

\section{Reviewing polynomials and lotteries}
See other \href{https://topos.site/blog/2023-03-23-distributions-and-lotteries/}{blog post}, pick interesting things. For example, the polynomial, what the monad multiplication means. Also, recall $\otimes$ and $[-,-]$ and what $[\lott,\yon]$ means.
%
%For any finite set $N$, define $\Delta_N$ to be the set of probability distributions on $N$, 
%
%\[ \Delta_N := \left \{  P: N \to [0,1] \mid  1 = \sum_{n:N} P_n \right \} \]
%
%\[ \lott = \sum_{N:\N} \sum_{p: \Delta_N} y^N  \]
%
%For any two polynomials $p = \sum_{P:p(1)}$ and $q$,
%\begin{align*}
% p \ox q &= \sum_{P: p(1)} \yon^{p[P]} \ox \sum_{Q: q(1)} \yon^{q[Q]} := \sum_{(P,Q): p(1) \times q(1)} \yon^{p[P] \times q[Q]} \\
% [p,q] &= \prod_{P:p(1)} \sum_{Q: q(1)} \prod_{e: q[Q]} \sum_{d:p[P]} 1 
% \end{align*}
%
%\begin{align}
%\nonumber
%\Poly(p,q) &= \Poly\left(\sum_{i:P} \yon^{p[P]}, q\right) \\\nonumber
%&\cong \prod_{P:p(1)} \Poly\left(\yon^{p[P]}, q\right) & \text{Universal property of coproduct}\\ \nonumber
%&\cong \prod_{P:p(1)} q(p[P]) &  \text{Yoneda Lemma} \\  \nonumber
%& \cong \prod_{P:p(1)} \sum_{Q: q(1)} p[P]^{q[Q]} \\\label{eqn.poly_map}
%& = \prod_{P:p(1)} \sum_{Q: q(1)} \prod_{e: q[Q]} \sum_{d:p[P]} 1 
%\end{align}

\section{Typical sequences and Martin-L\"of randomness}

\dnote{Priyaa, see \url{https://chatgpt.com/share/20a5704c-6c87-4f49-9b57-4e632c312e38}}

A $[\lott,\yon]$-coalgebra consists of a set $S$ and a function $S\to[\lott,\yon](S)$. We say that it is \emph{typical} if, for any $s\in S$ the sequence is typical. We say that it is \emph{algorithmically random} if, for any $s\in S$ the sequence is algorithmically random.




\section{The ``God machine''}

Albert Einstein's God did not play dice, but perhaps David Bohm's ``guiding wave'' does. Here we consider the idea of a machine, which Priyaa calls the "God machine", that picks a winning ticket from any named lottery. 

Construction: for each lottery $(N,P)$, a stream $s_{N,P}:\nn\to N$ that's typical, algorithmically random, etc. The coalgebra outputs these, and then for each actual choice $(N,P)$ of lottery played, it rips off one ticket. (Write this as a formula.)

This has the property that for any list $(N_1,P_1),\ldots(N_k,P_k)$ with each $(N_i,P_i)\neq (N,P)$, one has $(s.(N_1,P_1)\ldots(N_k,P_k))(N,P))=s(N,P)$, i.e.\ that playing one lottery cannot affect which ticket is poised to win another lottery. 

Prove that if $s$ is typical, then for any lottery $(N,P)$, the stream $s.(N,P)$ is also typical; same for algorithmically random. This means that typicality and algorithmic randomness define propositions in the internal logic of the topos $[\lott,\yon]\coalg$.

That is, you might say that the terminal coalgebra such that every element of it is typical is best!


\end{document}